\hypertarget{ux3c4ux3b1-ux3b5ux3c0ux3c4ux3b1ux3bdux3b7ux3c3ux3b1-ux3baux3b1ux3b9-ux3c4ux3bf-ux3b9ux3bfux3bdux3b9ux3bf-ux3c0ux3b1ux3bdux3b5ux3c0ux3b9ux3c3ux3c4ux3b7ux3bcux3b9ux3bf}{%
\chapter{ΤΑ ΕΠΤΑΝΗΣΑ ΚΑΙ ΤΟ ΙΟΝΙΟ
ΠΑΝΕΠΙΣΤΗΜΙΟ}\label{ux3c4ux3b1-ux3b5ux3c0ux3c4ux3b1ux3bdux3b7ux3c3ux3b1-ux3baux3b1ux3b9-ux3c4ux3bf-ux3b9ux3bfux3bdux3b9ux3bf-ux3c0ux3b1ux3bdux3b5ux3c0ux3b9ux3c3ux3c4ux3b7ux3bcux3b9ux3bf}}

Έχοντας διατελέσει υπό την κυριαρχία των Βενετών, των Γάλλων και των
Άγγλων, τα Επτάνησα διαφοροποιούνται ιστορικά από τις υπόλοιπες
ελληνικές επαρχίες, οι οποίες με την κατάλυση του Βυζαντινού Κράτους
κατά το 15ο αιώνα εντάχθηκαν στην Οθωμανική αυτοκρατορία μέχρι το 19ο
και τις αρχές του 20ού, οπότε άρχισε η σταδιακή απελευθέρωσή τους και η
ένταξή τους στο νεοελληνικό εθνικό κράτος. Κατ' αυτό τον τρόπο, τα Ιόνια
νησιά ήλθαν πλησιέστερα στους δυτικούς ευρωπαϊκούς τρόπους ζωής και
σκέψης, που με τη σειρά τους άφησαν το στίγμα τους στο αστικό και
αγροτικό τοπίο των νησιών, στις νοοτροπίες και στις συνήθειες των
ανθρώπων. Είναι ενδεικτικό ότι το πρώτο ελληνικό Πανεπιστήμιο, η Ιόνιος
Ακαδημία, ιδρύθηκε στην Κέρκυρα το 1824, κατά την περίοδο της
Αγγλοκρατίας. Από τις πρώτες δεκαετίες του 19ου αιώνα και μέχρι την
ένταξη των Επτανήσων στο νεοελληνικό κράτος το 1864 αναπτύχθηκε εκεί
αξιόλογο λογοτεχνικό ρεύμα με ιδιαίτερα χαρακτηριστικά, που προσέφερε
πολλά στις πνευματικές ζυμώσεις του σύγχρονου ελληνισμού. Κατά την
τελευταία εικοσαετία, το ελληνικό κράτος ίδρυσε νέα Πανεπιστήμια στην
περιφέρεια, με στόχο την πολιτιστική και οικονομική αναβάθμισή της, αλλά
και την προώθηση επιστημών που δεν περιλαμβάνονταν στα παλαιότερα ΑΕΙ.
Στην πολιτική αυτή εντάσσεται και η ίδρυση του Ιονίου Πανεπιστημίου, το
οποίο φιλοδοξεί, και σε μεγάλο βαθμό έχει επιτύχει, να επανασυνδεθεί με
την εντόπια πνευματική παράδοση, γέννημα των πολιτιστικών επιδράσεων που
είχε δεχθεί το νησί κατά τη μακρά περίοδο των ευρωπαϊκών κυριαρχιών που
γνώρισε.

Το Ιόνιο Πανεπιστήμιο Το Ιόνιο Πανεπιστήμιο ιδρύθηκε το 1984, με έδρα
την Κέρκυρα, μαζί με τα Πανεπιστήμια της Θεσσαλίας και του Αιγαίου. H
διασπορά των κτηριακών εγκαταστάσεων του Πανεπιστημίου σε διαφορετικά
σημεία της πόλης της Κέρκυρας έχει ως αποτέλεσμα την ένταξή του στις
χωροταξικές και κοινωνικές δομές της πόλης. Από το 2018, με την
ενσωμάτωση του Τ.Ε.Ι. Ιονίων Νήσων, το Ιόνιο Πανεπιστήμιο επεκτείνει τη
λειτουργία του με έξι νέα τμήματα, πέντε από αυτά σε άλλα τρία νησιά,
την Λευκάδα, την Κεφαλονιά και τη Ζάκυνθο. Δικτυακός Τόπος Ιονίου
Πανεπιστημίου: http://www.ionio.gr.

\#\#Σχολές και Τμήματα Το Ιόνιο Πανεπιστήμιο απαρτίζεται από τις
ακόλουθες Σχολές: Σχολή Ιστορίας και Μετάφρασης-Διερμηνείας, η οποία
περιλαμβάνει τα εξής τμήματα: • Τμήμα Ιστορίας • Τμήμα Ξένων Γλωσσών,
Μετάφρασης και Διερμηνείας Σχολή Μουσικής και Οπτικοακουστικών Σπουδών,
η οποία περιλαμβάνει τα εξής τμήματα: • Τμήμα Μουσικών Σπουδών • Τμήμα
Τεχνών Ήχου και Εικόνας • Τμήμα Εθνομουσικολογίας Σχολή Επιστήμης της
Πληροφορίας και Πληροφορικής, η οποία περιλαμβάνει τα εξής τμήματα: •
Τμήμα Aρχειονομίας, Bιβλιοθηκονομίας και Μουσειολογίας • Τμήμα
Πληροφορικής • Το Τμήμα Ψηφιακών Μέσων και Επικοινωνίας Σχολή
Περιβάλλοντος, η οποία περιλαμβάνει τα εξής τμήματα: • Τμήμα
Περιβάλλοντος • Τμήμα Επιστήμης και Τεχνολογίας Τροφίμων Σχολή
Οικονομικών Επιστημών, η οποία περιλαμβάνει τα εξής τμήματα: • Τμήμα
Περιφερειακής Ανάπτυξης • Τμήμα Τουρισμού
