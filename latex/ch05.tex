\hypertarget{ux3b5ux3c1ux3b5ux3c5ux3bdux3b1-ux3c3ux3c4ux3bf-ux3c4ux3bcux3b7ux3bcux3b1-ux3c0ux3bbux3b7ux3c1ux3bfux3c6ux3bfux3c1ux3b9ux3baux3b7ux3c3}{%
\chapter{ΕΡΕΥΝΑ ΣΤΟ ΤΜΗΜΑ
ΠΛΗΡΟΦΟΡΙΚΗΣ}\label{ux3b5ux3c1ux3b5ux3c5ux3bdux3b1-ux3c3ux3c4ux3bf-ux3c4ux3bcux3b7ux3bcux3b1-ux3c0ux3bbux3b7ux3c1ux3bfux3c6ux3bfux3c1ux3b9ux3baux3b7ux3c3}}

\hypertarget{ux3b8ux3b5ux3c3ux3bcux3bfux3b8ux3b5ux3c4ux3b7ux3bcux3adux3bdux3b1-ux3b5ux3c1ux3b5ux3c5ux3bdux3b7ux3c4ux3b9ux3baux3ac-ux3b5ux3c1ux3b3ux3b1ux3c3ux3c4ux3aeux3c1ux3b9ux3b1}{%
\section{Θεσμοθετημένα Ερευνητικά
Εργαστήρια}\label{ux3b8ux3b5ux3c3ux3bcux3bfux3b8ux3b5ux3c4ux3b7ux3bcux3adux3bdux3b1-ux3b5ux3c1ux3b5ux3c5ux3bdux3b7ux3c4ux3b9ux3baux3ac-ux3b5ux3c1ux3b3ux3b1ux3c3ux3c4ux3aeux3c1ux3b9ux3b1}}

Στο Τμήμα Πληροφορικής, λειτουργούν τα ακόλουθα θεσμοθετημένα
εργαστήρια:

\begin{itemize}
\tightlist
\item
  \emph{Εργαστήριο Βιοπληροφορικής και Ανθρώπινης Ηλεκτροφυσιολογίας
  (BiHELab)}
\item
  \emph{Εργαστήριο Πληροφοριακών Συστημάτων και Βάσεων Δεδομένων
  (ISDLab)}
\item
  \emph{Εργαστήριο Δικτύων, Πολυμέσων, και Ασφάλειας Συστημάτων
  (NMSLab)}
\item
  \emph{Εργαστήριο Εφαρμογών Πληροφορικής στις Ανθρωπιστικές --
  Κοινωνικές Επιστήμες (HILab)}
\item
  \emph{Εργαστήριο Υπολογιστικής Μοντελοποίησης (CMODLab)}
\end{itemize}

\hypertarget{ux3b5ux3c1ux3b3ux3b1ux3c3ux3c4ux3aeux3c1ux3b9ux3bf-ux3b2ux3b9ux3bfux3c0ux3bbux3b7ux3c1ux3bfux3c6ux3bfux3c1ux3b9ux3baux3aeux3c2-ux3baux3b1ux3b9-ux3b1ux3bdux3b8ux3c1ux3ceux3c0ux3b9ux3bdux3b7ux3c2-ux3b7ux3bbux3b5ux3baux3c4ux3c1ux3bfux3c6ux3c5ux3c3ux3b9ux3bfux3bbux3bfux3b3ux3afux3b1ux3c2-bihelab}{%
\subsubsection{Εργαστήριο Βιοπληροφορικής και Ανθρώπινης
Ηλεκτροφυσιολογίας
(BIHELab)}\label{ux3b5ux3c1ux3b3ux3b1ux3c3ux3c4ux3aeux3c1ux3b9ux3bf-ux3b2ux3b9ux3bfux3c0ux3bbux3b7ux3c1ux3bfux3c6ux3bfux3c1ux3b9ux3baux3aeux3c2-ux3baux3b1ux3b9-ux3b1ux3bdux3b8ux3c1ux3ceux3c0ux3b9ux3bdux3b7ux3c2-ux3b7ux3bbux3b5ux3baux3c4ux3c1ux3bfux3c6ux3c5ux3c3ux3b9ux3bfux3bbux3bfux3b3ux3afux3b1ux3c2-bihelab}}

http://bihelab.di.ionio.gr

Το εργαστήριο Βιοπληροφορικής και Ανθρώπινης Ηλεκτροφυσιολογίας καλύπτει
τις ερευνητικές δραστηριότητες του Τμήματος Πληροφορικής του Ιονίου
Πανεπιστημίου, σχετικά με τις νευρολογικές διαταραχές και τη συσχέτιση
τους με τις υποκυτταρικές μετρήσεις βιοενέργεια.Οι νευροεκφυλιστικές
παθήσεις, προσβάλλουν πλέον ένα σημαντικό ποσοστό του πληθυσμού των
σύγχρονων δυτικών κοινωνιών. Τα τελευταία χρόνια καταβάλλεται μια
συστηματική προσπάθεια αποσαφήνισης των παθογενετικών παραγόντων αυτών
των νοσημάτων, τα οποία πιστεύεται ότι, ανεξάρτητα από τη
συμπτωματολογία, σε ένα μεγάλο βαθμό ακολουθούν κοινούς μηχανισμούς
παθογένεσης.Βασικός στόχος του εργαστηρίου είναι η καταγραφή νέων και
αποτελεσματικών πρωτοκόλλων διάγνωσης διαφόρων τύπων άνοιας και
συγκεκριμένα νευρολογικών διαταραχών μέσα από τον εντοπισμό, τη
χαρτογράφηση, τη βιολογική ανάλυση καθώς και τη μαθηματική μοντελοποίηση
και προσομοίωση όλων των παραγόντων που σχετίζονται με τις
μιτοχονδριακές δυσλειτουργίες, έτσι ώστε σύντομα να βελτιωθούν οι
υφιστάμενες τεχνικές αντιμετώπισης τους αλλά και να δημιουργηθούν στο
μέλλον νέες στοχευμένες θεραπείες.Ο εξοπλισμός του εργαστηρίου,
περιλαμβάνει εξειδικευμένα μηχανήματα βιολογικών αναλύσεων, συστήματα
για τη συλλογή και ανάλυση δεδομένων απεικόνισης, μικροσκόπια, κάμερες
υψηλής ανάλυσης, συσκευές μέτρησης ανθρώπινης ηλεκτροφυσιολογίας,
συσκευές μέτρησης πληθυσμού σωματιδίων καθώς και υποστηρικτικά
λογισμικά.

\hypertarget{ux3b5ux3c1ux3b3ux3b1ux3c3ux3c4ux3aeux3c1ux3b9ux3bf-ux3c0ux3bbux3b7ux3c1ux3bfux3c6ux3bfux3c1ux3b9ux3baux3ceux3bd-ux3c3ux3c5ux3c3ux3c4ux3b7ux3bcux3acux3c4ux3c9ux3bd-ux3baux3b1ux3b9-ux3b2ux3acux3c3ux3b5ux3c9ux3bd-ux3b4ux3b5ux3b4ux3bfux3bcux3adux3bdux3c9ux3bd-isdlab}{%
\subsubsection{Εργαστήριο Πληροφορικών Συστημάτων και Βάσεων Δεδομένων
(ISDLab)}\label{ux3b5ux3c1ux3b3ux3b1ux3c3ux3c4ux3aeux3c1ux3b9ux3bf-ux3c0ux3bbux3b7ux3c1ux3bfux3c6ux3bfux3c1ux3b9ux3baux3ceux3bd-ux3c3ux3c5ux3c3ux3c4ux3b7ux3bcux3acux3c4ux3c9ux3bd-ux3baux3b1ux3b9-ux3b2ux3acux3c3ux3b5ux3c9ux3bd-ux3b4ux3b5ux3b4ux3bfux3bcux3adux3bdux3c9ux3bd-isdlab}}

http://isdlab.di.ionio.gr

Το Εργαστήριο Πληροφορικών Συστημάτων και Βάσεων Δεδομένων (ISDLab)
ιδρύθηκε τον Δεκέμβριο 2015, παρόλο που λειτουργούσε ήδη ατύπως από τον
Σεπτέμβριο 2011 με τον διακριτικό τίτλο DBISLab με συμμετοχή μελών του
σε Επιστημονικά Συνέδρια αλλά και πολλές δημοσιεύσεις.Οι επιστημονικές
περιοχές που το εργαστήριο κυρίως καλύπτει αφορούν στις ακόλουθες:-
ανάλυση και σχεδιασμό πληροφοριακών συστημάτων,

\begin{itemize}
\item
  αξιολόγηση καινοτόμων εφαρμογών πληροφορικής,
\item
  ηλεκτρονικό εμπόριο και ηλεκτρονικό επιχειρείν,
\item
  ηλεκτρονική διακυβέρνηση, συστήματα βάσεων δεδομένων,
\item
  συστήματα data warehousing και datamining,
\item
  συστήματα data stream management,
\item
  συστήματα διαχείρισης μεγάλου όγκου δεδομένων,
\item
  διαχείριση δεδομένων στον παγκόσμιο ιστό,
\item
  συστήματα υπολογιστικού νέφους,
\item
  δίκτυο−κεντρικά πληροφοριακά συστήματα και συναφή επιστημονικά
  αντικείμενα.
\end{itemize}

Το Εργαστήριο Πληροφορικών Συστημάτων και Βάσεων Δεδομένων (ISDLab) έχει
στελεχωθεί από καθηγητές του Τμήματος Πληροφορικής και λοιπό τεχνικό και
επιστημονικό προσωπικό. Έχει ως αποστολή:- Την κάλυψη σε προπτυχιακό και
μεταπτυχιακό επίπεδο των διδακτικών και ερευνητικών αναγκών του Τμήματος
Πληροφορικής καθώς και των άλλων τμημάτων του Ιονίου Πανεπιστημίου, σε
θέματα που εμπίπτουν στα αντικείμενα δραστηριότητας του εργαστηρίου όπως
αυτά προσδιορίζονται στο άρθρο 1 του ΦΕΚ Ίδρυσης του (ΦΕΚ Ίδρυσης: τ.Β
Αρ. Φυλλου 2616 -- 04.12.2015).

\begin{itemize}
\item
  Την ανάπτυξη προγραμμάτων διδασκαλίας και τη διεξαγωγή βασικής και
  εφαρμοσμένης έρευνας.
\item
  Τη συνεργασία κάθε μορφής με κέντρα ερευνών και ακαδημαϊκά ιδρύματα
  ελληνικά και αλλοδαπά, εφόσον οι επιστημονικοί στόχοι, συμπίπτουν,
  συμβαδίζουν και αλληλοσυμπληρώνονται με εκείνους του εργαστηρίου.
\item
  Τη διοργάνωση επιστημονικών διαλέξεων, ημερίδων, σεμιναρίων,
  συμποσίων, συνεδρίων και άλλων επιστημονικών εκδηλώσεων, την
  πραγματοποίηση δημοσιεύσεων και εκδόσεων και την πρόσκληση Ελλήνων και
  ξένων αναγνωρισμένων επιστημόνων.
\item
  Την εκπόνηση επιστημονικών μελετών συναφών με το αντικείμενο του
  εργαστηρίου.
\item
  Την παροχή υπηρεσιών σε ιδιώτες και σε κάθε νομικής μορφής οργανισμούς
  κατά τα προβλεπόμενα στο Π.δ. 159/1984 (Α΄/53).
\end{itemize}

Το Εργαστήριο Πληροφορικών Συστημάτων και Βάσεων Δεδομένων (ISDLab) έχει
στελεχωθεί από καθηγητές του Τμήματος Πληροφορικής και λοιπό τεχνικό και
επιστημονικό προσωπικό.

\hypertarget{ux3b5ux3c1ux3b3ux3b1ux3c3ux3c4ux3aeux3c1ux3b9ux3bf-ux3b4ux3b9ux3baux3c4ux3cdux3c9ux3bd-ux3c0ux3bfux3bbux3c5ux3bcux3adux3c3ux3c9ux3bd-ux3baux3b1ux3b9-ux3b1ux3c3ux3c6ux3acux3bbux3b5ux3b9ux3b1ux3c2-ux3c3ux3c5ux3c3ux3c4ux3b7ux3bcux3acux3c4ux3c9ux3bd-nmslab}{%
\subsubsection{Εργαστήριο Δικτύων, Πολυμέσων και Ασφάλειας Συστημάτων
(NMSLab)}\label{ux3b5ux3c1ux3b3ux3b1ux3c3ux3c4ux3aeux3c1ux3b9ux3bf-ux3b4ux3b9ux3baux3c4ux3cdux3c9ux3bd-ux3c0ux3bfux3bbux3c5ux3bcux3adux3c3ux3c9ux3bd-ux3baux3b1ux3b9-ux3b1ux3c3ux3c6ux3acux3bbux3b5ux3b9ux3b1ux3c2-ux3c3ux3c5ux3c3ux3c4ux3b7ux3bcux3acux3c4ux3c9ux3bd-nmslab}}

http://nmslab.di.ionio.gr

Το Εργαστήριο Δικτύων, Πολυμέσων και Ασφάλειας Συστημάτων (NMSLab)
υποστηρίζει τις εκπαιδευτικές και ερευνητικές ανάγκες του Τμήματος
Πληροφορικής, καθώς και άλλων Τμημάτων του Ιονίου Πανεπιστημίου, σχετικά
με τις γνωστικές περιοχές δίκτυα υπολογιστών, πολυμέσα και ασφάλεια
πληροφοριών.Ο εκπαιδευτικός ρόλος του NMSLab είναι να υποστηρίζει τα
μαθήματα του προπτυχιακού προγράμματος σπουδών που σχετίζονται με δίκτυα
υπολογιστών, συστήματα πολυμέσων σχετικά με τον πολιτισμό, ασφάλεια
υπολογιστών, κρυπτογραφία, ασφάλεια πληροφοριακών συστημάτων,
πληροφοριακή ιδιωτικότητα, θεωρία πληροφοριών. Αναφορικά με μαθήματα
μεταπτυχιακών προγραμμάτων σπουδών υποστηρίζει παρόμοια μαθήματα με
έμφαση στην έρευνα, καινοτόμα ερευνητικά πεδία και προκλήσεις της
γνωστικής περιοχής.Οι κύριες περιοχές έμφασης της έρευνας σχετικά με
δίκτυα υπολογιστών είναι:- Ad-hoc δίκτυα

\begin{itemize}
\item
  Ασύρματα δίκτυα αισθητήρων
\item
  Δίκτυα νεφοϋπολογιστικής
\item
  Κατανεμημένα και κινητά συστήματα
\end{itemize}

Οι κύριες περιοχές σχετικά με έρευνα σε πολυμέσα αφορούν σε
τρισδιάστατες αναπαραστάσεις και σχετικές τεχνικές που συλλαμβάνουν τις
ιδιοσυγκρασίες που έχουν φυσικά και ανθρωπίνως κατασκευασμένα
περιβάλλοντα για:- Διατήρηση πολιτισμικής και φυσικής κληρονομίας

\begin{itemize}
\item
  Προώθηση τοπικής κουλτούρας
\item
  Ανάπτυξης εκπαιδευτικών εφαρμογών
\item
  Προαγωγή νέων τεχνολογιών εικονικού κόσμου
\end{itemize}

Στον τομέα της ασφάλειας συστημάτων η έρευνα επικεντρώνεται στη χρήση
μέτρων ασφάλεαις (π.χ. κρυπτογραφικών τεχνικών) για την προστασία της
ασφάλειας πληροφοριών και της πληροφοριακής ιδιωτικότητας.
Συγκεκριμένα:- Ασφάλεια και ιδιωτικότητα σε οχηματικά δίκτυα

\begin{itemize}
\item
  Εξόρυξη δεδομένων διατηρώντας την ιδιωτικότητα
\item
  Ασφάλεια και ιδιωτικότητα σε εφαρμογές βασισμένες στην τοποθεσία
\item
  Ανάλυση και διαχείριση επικινδυνότητας
\item
  Πολιτικές ασφάλειας
\item
  Ασφάλεια επικοινωνιών σε κατανεμημένα δίκτυα αισθητήρων
\end{itemize}

\hypertarget{ux3b5ux3c1ux3b3ux3b1ux3c3ux3c4ux3aeux3c1ux3b9ux3bf-ux3b5ux3c6ux3b1ux3c1ux3bcux3bfux3b3ux3ceux3bd-ux3c0ux3bbux3b7ux3c1ux3bfux3c6ux3bfux3c1ux3b9ux3baux3aeux3c2-ux3c3ux3c4ux3b9ux3c2-ux3b1ux3bdux3b8ux3c1ux3c9ux3c0ux3b9ux3c3ux3c4ux3b9ux3baux3adux3c2-ux3baux3bfux3b9ux3bdux3c9ux3bdux3b9ux3baux3adux3c2-ux3b5ux3c0ux3b9ux3c3ux3c4ux3aeux3bcux3b5ux3c2-hilab}{%
\subsubsection{Εργαστήριο Εφαρμογών Πληροφορικής στις Ανθρωπιστικές --
Κοινωνικές Επιστήμες
(HILab)}\label{ux3b5ux3c1ux3b3ux3b1ux3c3ux3c4ux3aeux3c1ux3b9ux3bf-ux3b5ux3c6ux3b1ux3c1ux3bcux3bfux3b3ux3ceux3bd-ux3c0ux3bbux3b7ux3c1ux3bfux3c6ux3bfux3c1ux3b9ux3baux3aeux3c2-ux3c3ux3c4ux3b9ux3c2-ux3b1ux3bdux3b8ux3c1ux3c9ux3c0ux3b9ux3c3ux3c4ux3b9ux3baux3adux3c2-ux3baux3bfux3b9ux3bdux3c9ux3bdux3b9ux3baux3adux3c2-ux3b5ux3c0ux3b9ux3c3ux3c4ux3aeux3bcux3b5ux3c2-hilab}}

http://di.ionio.gr/en/uncategorized/humanistic-and-social-informatics-lab-hilab-2/hilab

Η Πληροφορική στις Ανθρωπιστικές και Κοινωνικές Επιστήμες εστιάζει αφ'
ενός στην εφαρμογή των ΤΠΕ στην εξαγωγή, αναπαράσταση και επεξεργασία
ανθρωπιστικών και κοινωνικών δεδομένων, όπως αυτά προέρχονται από
αντίστοιχες επιστήμες και τέχνες, όπως η Ψυχολογία, η Γλωσσολογία, η
Ιστορία, η Αρχαιολογία, η Φιλοσοφία, η Ανθρωπολογία, η Κοινωνιολογία, η
Μουσική, οι Καλές και οι Εφαρμοσμένες Τέχνες.Αφ' ετέρου χρησιμοποιεί
μεθοδολογίες των επιστημών αυτών για να προσδώσει στις υπηρεσίες και τα
προϊόντα των νέων τεχνολογιών βέλτιστη ποιότητα και ανθρωποκεντρική
διάσταση.Το Εργαστήριο Εφαρμογών Πληροφορικής στις Ανθρωπιστικές και
Κοινωνικές Επιστήμες δραστηριοποιείται στην εκπαίδευση, την έρευνα και
την ανάπτυξη στις περιοχές της Πολιτιστικής Πληροφορικής, της Μουσικής
Πληροφορικής, της Ιστορικής Πληροφορικής, της Υπολογιστικής Γλωσσολογίας
και της Επεξεργασίας Φυσικής Γλώσσας, του Ψυχαγωγικού Λογισμικού, των
Πολυμέσων, της Εικονικής Πραγματικότητας, της Επεξεργασίας Εικόνας, των
Εφαρμογών Νέων Τεχνολογιών στην Εκπαίδευση, της Αναπαράστασης και
Διαχείρισης Γνώσης, των Συνεργατικών Συστημάτων, των Ψηφιακών Μέσων.

\hypertarget{ux3b5ux3c1ux3b3ux3b1ux3c3ux3c4ux3aeux3c1ux3b9ux3bf-ux3c5ux3c0ux3bfux3bbux3bfux3b3ux3b9ux3c3ux3c4ux3b9ux3baux3aeux3c2-ux3bcux3bfux3bdux3c4ux3b5ux3bbux3bfux3c0ux3bfux3afux3b7ux3c3ux3b7ux3c2-cmodlab}{%
\subsubsection{Εργαστήριο Υπολογιστικής Μοντελοποίησης
(CMODLab)}\label{ux3b5ux3c1ux3b3ux3b1ux3c3ux3c4ux3aeux3c1ux3b9ux3bf-ux3c5ux3c0ux3bfux3bbux3bfux3b3ux3b9ux3c3ux3c4ux3b9ux3baux3aeux3c2-ux3bcux3bfux3bdux3c4ux3b5ux3bbux3bfux3c0ux3bfux3afux3b7ux3c3ux3b7ux3c2-cmodlab}}

http://cmodlab.di.ionio.gr

Το Εργαστήριο Υπολογιστικής Μοντελοποίησης διεξάγει θεωρητική έρευνα και
επιτελεί εφαρμοσμένη ερευνητική εργασία για την ανάπτυξη, υιοθέτηση και
διαχείριση καινοτόμων εφαρμογών δια μέσου Μαθηματικών Μοντέλων και
Προσομοιώσεων, οι οποίες θα οδηγήσουν στη διαμόρφωση και προβολή ενός
ελκυστικού και ανταγωνιστικού αναπτυξιακού περιβάλλοντος της σύγχρονης
Ελληνικής Κοινωνίας και Επιστημονικής Κοινότητας.Οι ερευνητικές
δραστηριότητες που υποστηρίζονται από το Εργαστήριο Υπολογιστικής
Μοντελοποίησης αφορούν ένα ευρύτατο φάσμα της επιστήμης της Πληροφορικής
με έμφαση στην ανάπτυξη μαθηματικών και υπολογιστικών τεχνικών για τη
μοντελοποίηση και προσομοίωση φυσικών (παράλληλων και κατανεμημένων)
συστημάτων. Συγκεκριμένα, στόχος των εν λόγω δραστηριοτήτων είναι αφενός
η διακριτοποίηση φυσικών νόμων και η θεμελίωση διακριτών γεωμετριών για
την αριθμητική περιγραφή φυσικών συστημάτων με συμβατό τρόπο
(διατηρώντας τις βασικές συμμετρίες) και αφετέρου η προσομοίωση και
κατανόηση κρίσιμων φαινομένων κυρίως σε σχέση με πολύπλοκα δίκτυα και
δυναμικές και εξελικτικές διεργασίες που λαμβάνουν χώρα σε αυτά.Οι
βασικοί επιστημονικοί και τεχνολογικοί κλάδοι που συνθέτουν την
τεχνογνωσία του εργαστηρίου είναι κυρίως οι εξής: Αναγνώριση Προτύπων,
Τεχνητή Νοημοσύνη/ Τεχνητά Νευρωνικά Δίκτυα, Ανάλυση αλγορίθμων (κυρίως
Γενετικοί αλγόριθμοι), Έμπειρα/ Ευφυή Συστήματα, Τεχνολογίες Διαχείρισης
Γνώσης, Τεχνολογίες Επεξεργασίας Ψηφιακού Ήχου και Εικόνας, Εικονική/
Επαυξημένη Πραγματικότητα, Τεχνολογίες και Μηχανική Λογισμικού,
Τεχνολογίες Προσωποποιημένης Αλληλεπίδρασης και Τεχνολογίες Αυτόματης
Επαλήθευσης και Σχεδίασης Συστημάτων.Πεδία εφαρμογής των παραπάνω, στα
πλαίσια των δραστηριοτήτων του εργαστηρίου περιλαμβάνουν τους εξής
τομείς:- Βιοπληροφορική

\begin{itemize}
\item
  Καινοτόμες μεταφορές αλληλεπίδρασης σε επιλεγμένα θεματικά πεδία (π.χ.
  εκπαιδευτικά περιβάλλοντα)
\item
  Συστήματα Βιομετρικής (Biometrics)
\item
  Ενσωματωμένα (embedded) Συστήματα Πραγματικού Χρόνου
\item
  Ανάπτυξη Υπολογιστικών Εφαρμογών για Τυχαία Συστήματα
\item
  Επεξεργασία Σήματος και Εικόνας
\item
  Υπολογιστικά Πλέγματα
\end{itemize}
